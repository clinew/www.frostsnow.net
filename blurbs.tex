\documentclass{article}

\usepackage{html}

\begin{document}

Thoughts that are not worth their own blog.

\subsection{2024-03-25}
As much as I'm liking my \htmladdnormallink{NexDock 360}{https://shop.puri.sm/shop/lapdock-kit/}, they sure make finding a replacement battery difficult.
NexDock has \htmladdnormallink{instructions for replacing a battery}{https://nexdock.com/support/nexdock-360-battery-replacement/} but nowhere to order one that I can see.
The information labels on the back of the battery in the aforementioned instructions show a Model #30154200P, 7.6V, 5800mAh and a "Carried" (Carrier?) Standard of GB31241.
Based on this I found \htmladdnormallink{a similar battery on AliExpress}{https://www.aliexpress.us/item/3256803233725202.html} though it's only 5000mAh and its "Carried" (Carrier?!) Standard is 18287-2013.
Hopefully it works, but it will take about two months to ship (highlighting the importance of always having a spare battery \emph{on hand}) and it will probably be longer than that before the inevitable happens and I try using the spare.

\subsection{2023-07-30}
I created an eBay account in order to buy something.
Next day my account was permanently suspended.
No reason given.
I used Web chat; they refused to provide a reason and closed the chat.
I called; they refused to provide a reason and hung up.
I sent them a physical letter; months later I have not heard back.
This makes eBay the single worst company I have ever dealt with.
I wouldn't recommend eBay to anyone.

\subsection{2023-02-12}
So I had this \htmladdnormallink{weird bug}{https://github.com/beagleboard/linux/issues/278} which turned out to have been caused by \texttt{umask} being set to \texttt{0777}.  Since all the documentation I found stated that the default value should be \texttt{0022}, fixing should be a matter of finding the configuration file where it was erroneously being set and editing it, right?  Was it set in \texttt{/etc/profile}?  No.  How about \texttt{/etc/profile.d}?  No.  \texttt{\textasciitilde /.bash_profile}?  No.  \texttt{\textasciitilde /.bashrc}?  No.  \texttt{/etc/bash/bashrc}?  \texttt{/etc/skel/.bash_profile}?  \texttt{/etc/skel/.bashrc}?  \texttt{/etc/profile.env}?  \texttt{/etc/env.d}?  \texttt{/etc/environment}?  No.  No.  No.  No.  No.  No.  Ugh.  I just set it at the end of \texttt{/etc/profile} for now.

\subsection{2023-02-05}
Had an issue building LibreCMC v1.5.12 where the build appeared to complete but no upgrade image was generated.  Re-running \texttt{make} with \texttt{V=s} showed: \texttt{[mktplinkfw] *** error: images are too big by 53697 bytes}.  I was able to fix this by running \texttt{make menuconfig} and disabling IPv6.  No one likes IPv6 anyways.

\subsection{2023-01-16}
I ran into an \htmladdnormallink{issue}{https://gogs.librecmc.org/libreCMC/libreCMC/issues/158} building LibreCMC v1.5.10 where none of the package mirrors for OpenSSL 1.1.1p had the package.  I was able to workaround this by opening \texttt{package/libs/openssl/Makefile} and adding a mirror from \htmladdnormallink{the OpenSSL mirror list}{https://www.openssl.org/source/mirror.html} to the \texttt{PKG_SOURCE_URL} variable.

\subsection{2022-12-26}
Had a Minetest mod issue that wasn't giving me a stack trace.  In order to get the stack trace I had to switch from Lua 5.1 to LuaJIT (\texttt{USE=-lua_single_target_lua_5-1 lua_single_target_luajit}).  Weird.

\subsection{2022-09-01}
I (eventually) managed to get \htmladdnormallink{Phasmophobia}{https://store.steampowered.com/app/739630/phasmophobia/} working.  Experimental Proton didn't work at all, but I got things sort of working with Proton 7.0-4.  First issue I ran into was:
\begin{verbatim}
002d:fixme:dbghelp:elf_search_auxv can't find symbol in module
\end{verbatim}
\ldots and much more followed by a crash.  This was fixed by setting the Steam launch command to \verb|PROTON_USE_WINED3D=1 %command%|.  Next I ran into an issue where the game would crash shortly after loading a contract with many messages of the form:
\begin{verbatim}
eventfd: Too many open files
\end{verbatim}
I was able to work around this by increasing the open file descriptor limit with the following series of commands:
\begin{verbatim}
su
ulimit -n 65536
su -l <user>
export DISPLAY=:0
\end{verbatim}
That being said, audio did not work at all.  That might have been because I was using ALSA and \texttt{apulse} rather than PulseAudio proper.  Trying to launch the game through Windows Steam via WINE, I first ran into an issue where, before I could type in two characters of my password, Steam would crash with the following error message:
\begin{verbatim}
Unhandled exception: unimplemented function urlmon.dll.414 called in 32-bit code (0x7bc51bb1).
\end{verbatim}
This was solved by running \texttt{winetricks urlmon}.  Though Steam would launch, it would display a black screen; I worked around this by launching the game directly:
\begin{verbatim}
wine "/path/to/steam.exe" -no-browser +open open/minigameslist -applaunch 739630
\end{verbatim}
Sometimes I get issues where the mouse/keyboard aren't picked up and/or released properly, but I can workaround that by a combination re-launching the game and re-loading \texttt{i3} as needed.  Other than that, the game, including multiplayer and voice recognition, have been running just fine.

\subsection{2022-06-13}
Had an issue with my Librem5 where calls would instantly disconnect.  Modem manager logs showed:
\begin{verbatim}
	ModemManager[647]: <warn>  [modem2] network reject indication received
	ModemManager[647]: <warn>  [modem2]   service domain: cs
	ModemManager[647]: <warn>  [modem2]   radio interface: lte
	ModemManager[647]: <warn>  [modem2]   reject cause: cs-service-temporarily-not-available
\end{verbatim}
Turns out the modem firmware was old.  Purism sent me a new one and calls now work!  Comparing the before and after output of \texttt{mmcli -m any} shows that, beyond IDs and signal quality, the only differences appear to be:
\begin{verbatim}
	firmware revision: MPSS.JO.2.0.2.c1.1-00032-9607_GENNS_PACK-1  1  [Feb 25 2019 01:00:00]
	   carrier config: default
\end{verbatim}
to
\begin{verbatim}
	      firmware revision: MPSS.JO.2.0.2.c1.1-00032-9607_GENNS_PACK-1.351938.1  1  [Nov 26 2020 02:00:00]
	         carrier config: ROW_Generic_3GPP
	carrier config revision: 05010821
\end{verbatim}
Not much to go on.  At least it's working now.

\subsection{2022-03-31}
Got this completely baffling error when running \texttt{repoman commit}:
\begin{verbatim}
<tmp/tmpva6xwekm.repoman.msg/COMMIT_EDITMSG" 4L, 150B written
* 0 files being committed...
error: gpg failed to sign the data
fatal: failed to write commit object
!!! Exiting on git (shell) error code: 128
\end{verbatim}
Yet \texttt{gpg -k} didn't reveal anything; it wasn't until I ran \texttt{touch ugh; gpg -v --clear-sign ugh} that I got \texttt{gpg: Note: signature key 52922D5D15C76E7A expired Mon Feb 28 22:45:16 2022 PST}.  Apparently a subkey expiration.  Modifying the expiration time of the new subkey was a simple matter of: navigating to the master key location, \texttt{export GNUPGHOME=.}, \texttt{gpg --with-keygrip --list-key <FINGERPRINT>} in order to find out which file in \texttt{private-keys-v1.d} \emph{actually} corresponded to the key fingerprint, \texttt{gpg --edit-key <FINGERPRINT>}, then, in the interactive prompt, \texttt{>key #}, \texttt{>expire}, \texttt{1y}, \texttt{>save} to close the interactive prompt, then \texttt{gpg --export <FINGERPRINT> > key.gpg} to export the key, copying the updated subkey from the offline location to the online location, and finally \texttt{gpg --import key.gpg}.  Simple.  Yeah.

\subsection{2022-02-06}
Had some trouble validating the signature on the newest \htmladdnormallink{TAILS}{https://tails.boum.org} image.  Apparently the new image is signed using a new subkey with fingerprint \texttt{753F901377A309F2731FA33F7BFBD2B902EE13D0}, but this subkey isn't visible on \texttt{hkp://pgp.surf.nl} and I had to manually retrieve it from a different keyserver with \texttt{gpg \verb|--|keyserver hkp://keyserver.ubuntu.com \verb|--|search-keys A490D0F4D311A4153E2BB7CADBB802B258ACD84F}.  Weird.

\subsection{2022-02-01}
I found a fix for slow UDP scans via \texttt{nmap} on systems which I have root on: relax the Linux kernel's ICMP ratelimiting.  Based on the info in \htmladdnormallink{ip-sysctl.txt}{https://www.kernel.org/doc/Documentation/networking/ip-sysctl.txt}, remove the ratelimit for ICMP ``Destination Unreachable'' messages by unmasking significant bit 3 in \texttt{/proc/sys/net/ipv4/icmp_ratemask} (default: \texttt{6168}) and also increase the overall message rate in \texttt{/proc/sys/net/ipv4/icmp_msgs_per_sec} from \texttt{1000} to \texttt{1000000}.  Restore old settings when finished scanning.  Reduced scan time from 2 days to a matter of minutes.

\subsection{2022-01-23}
I had some more fun upgrading from PiHole v5.2.1 to v5.8.1 (IIRC; I forgot to write the version numbers down).  First up was \texttt{FTL failed to start due to cannot open or create lease file /var/lib/misc/dnsmasq.leases: Permission denied} which was fixed by a change of ownership: \texttt{chown pihole:pihole /var/lib/misc}.  Next was a more perplexing \texttt{Warning in dnsmasq core: no address range available for DHCP request via lo}; I ended up fixing this via the PiHole Web interface by navigating to \texttt{Settings -> DNS} and then setting \texttt{Bind only to interface eth0}.  Not exactly sure what PiHole was trying to do there, but it seems to be functional now.

\subsection{2022-01-17}
I decided to upgrade my old \htmladdnormallink{Think Penguin router}{https://www.thinkpenguin.com/gnu-linux/free-software-wireless-n-broadband-router-gnu-linux-tpe-nwifirouter2} with the latest \htmladdnormallink{LibreCMC}{https://librecmc.org/index.html} version.  The image I needed wasn't available; so I braced myself for a difficult cross-compilation journey.  Thankfully, it turned out to be a smooth process of: downloading \htmladdnormallink{the source}{https://gogs.librecmc.org/libreCMC/libreCMC/archive/v1.5.7-20211001.tar.gz}, running \texttt{make menuconfig}, setting \texttt{Subtarget} to \texttt{Devices with small flash}, setting \texttt{Target Profile} to \texttt{TP-Link TL-WR841N/ND v8}, then running \texttt{make V=s}.  After a few hours and 7.5GB of disk space I had an upgrade image at \texttt{bin/targets/ath79/tiny/librecmc-ath79-tiny-tplink_tl-wr841-v8-squashfs-sysupgrade.bin} which worked as expected (the upgrading command is posted in a previous blurb).  My compliments to the author!

\subsection{2021-07-04}
Finally figured out my mic issue on \htmladdnormallink{FSF Jitsi Meet}{http://jitsi.member.fsf.org/}.  Apparently there's a difference between audio input options ``Monitor of Built-in Audio Analog Stereo'' and ``Built-in Audio Analog Stereo'', but once a microphone has been allowed there's no way to even \emph{view} or modify the selection without manually \emph{revoking} the permission, refreshing the page, and selecting a different option to allow.  This is a good example of the distinction between \emph{hiding} complexity rather than \emph{managing} complexity.

\subsection{2021-03-28}
Some quick notes on setting up \htmladdnormallink{Pi-Hole}{https://discourse.pi-hole.net} on a \htmladdnormallink{BeagleBone Black}{https://beagleboard.org/black/} with a \htmladdnormallink{Debian image}{https://beagleboard.org/latest-images} installed.  I had to disable Nginx:
\begin{verbatim}
systemctl disable nginx
\end{verbatim}
I fixed the following \texttt{dnsmasq} issues:
\begin{verbatim}
FTL failed to start due to illegal repeated keyword at line 8 of /etc/dnsmasq.d/SoftAp0
FTL failed to start due to cannot open or create lease file /var/run/dnsmasq.leases: Permission denied
\end{verbatim}
\ldots by opening \texttt{/etc/dnsmasq.d/SoftAp0} and removing both \texttt{cache-size=2048} and \texttt{dhcp-leasefile=/var/run/dnsmasq.leases} then opening \texttt{/etc/dnsmasq.d/01-pihole.conf} and adding \texttt{dhcp-leasefile=/var/run/pihole/leases}.  The changes to \texttt{SoftAp0} got reset on boot and even opening \texttt{/etc/default/bb-wl18xx} and changing \texttt{USE_GENERATED_DNSMASQ=yes} to \texttt{USE_GENERATED_DNSMASQ=no} didn't help, so I instead ran \texttt{chattr +i /etc/dnsmasq.d/SoftAp0}.  Lastly I ran \texttt{systemctl disable dnsmasq} which generates the following log message at boot (slightly edited) \texttt{FAILED Failed to start dnsmasq - ...t DHCP and caching DNS server.} but Pi-Hole seems to work fine anyways.

Lastly, to always boot from the SDCard I opened \texttt{/boot/uEnv.txt} on the eMMC filesystem and un-commented \texttt{disable_uboot_overlay_emmc=1} (thanks to \htmladdnormallink{this page}{https://elinux.org/Beagleboard:BeagleBoneBlack_Debian#U-Boot_Disable_on-board_devices}).  Things appear to be working now\ldots

\subsection{2021-03-01}
Thanks to \htmladdnormallink{this forum post}{https://bbs.archlinux.org/viewtopic.php?id=223241}, I learned that the workaround for:
\begin{verbatim}
gpg: key ID: public key "NAME (COMMENT) <EMAIL>" imported
gpg: key ID/SUBID: error sending to agent: Permission denied
gpg: error reading 'subkey.gpg': Permission denied
gpg: import from 'subkey.gpg' failed: Permission denied
gpg: Total number processed: 0
gpg:               imported: 1
gpg:       secret keys read: 1
\end{verbatim}
\ldots was to add \texttt{--pinentry-mode loopback}.

\subsection{2021-02-22}
I wrote a quick \htmladdnormallink{ebuild}{https://github.com/clinew/overlay_frostsnow/blob/master/games-server/valheim/valheim-9999.ebuild} for the \htmladdnormallink{Valheim}{https://www.valheimgame.com/} server.  No guarantees with regards to its efficacy.

\subsection{2020-11-09}
Glenn Greenwald, most famous for helping Snowden to expose the government's betrayal of U.S. citizens' Fourth Amendment rights, has \htmladdnormallink{resigned}{https://greenwald.substack.com/p/my-resignation-from-the-intercept} from The Intercept citing, ``repression, censorship, and ideological homogeneity''.  Fake News is real.

\subsection{2020-09-27}
I managed to get ``Borderlands 2'' working thanks to the good people on this \htmladdnormallink{Gentoo forums post}{https://forums.gentoo.org/viewtopic-t-1074698.html}.

\subsection{2020-08-25}
Should this blurb read, ``Facebook considered harmful'' or ``Oculus considered harmful''?  At this point, it \htmladdnormallink{makes no difference}{http://n-gate.com/hackernews/2020/08/21/0/}.

\subsection{2020-08-02}
Below are my brief notes on Gentoo auto-login with SSDM; it should be more secure than running \texttt{xorg-server} with the \texttt{suid} flag.  Create file \texttt{/etc/sddm.conf.d/autologin.conf} with contents:
\begin{verbatim}
[Autologin]
User=myuser
Session=i3.desktop
\end{verbatim}
In file \texttt{/etc/conf.d/xdm}, set the following line:
\begin{verbatim}
DISPLAYMANAGER="sddm"
\end{verbatim}
Then be sure to add \texttt{xdm} to the default runlevel with \texttt{rc-update add xdm default}.

\subsection{2020-07-14}
The Revolution will be corporate-sponsored.

\subsection{2020-07-01}
YouTube has updated their website to be a fetid morass of JavaScript.  Some good alternatives appear to be \htmladdnormallink{Invidious}{https://invidio.us/}, \htmladdnormallink{youtube-viewer}{https://github.com/trizen/youtube-viewer}, and \htmladdnormallink{straw-viewer}{https://github.com/trizen/straw-viewer}.

\subsection{2020-06-27}
``Bioshock: Infinite'' is starting to look less like a video game and more like a prophecy.

\subsection{2020-04-14}
Zoom \htmladdnormallink{considered harmful}{https://www.schneier.com/blog/archives/2020/04/security_and_pr_1.html}.

\subsection{2020-03-18}
Web \htmladdnormallink{considered harmful}{https://drewdevault.com/2020/03/18/Reckless-limitless-scope.html}.

\subsection{2020-02-03}
Last time I audited my home network with \texttt{nmap}, my scanning box got banned from my server, so this time I used \texttt{fail2ban-client set ssh-iptables addignoreip IP\_ADDRESS} in order to prevent my scanning box from getting banned.

\subsection{2020-01-22}
Got this error when trying to mount my BeagleBone Black's boot partition:
\begin{verbatim}
[883063.268285] FAT-fs (mmcblk0p1): IO charset ascii not found
\end{verbatim}
Turns out the proper kernel module was compressed for some reason and wouldn't load:
\begin{verbatim}
dante /lib/modules # insmod 4.14.71/kernel/fs/nls/nls_ascii.ko.xz
insmod: ERROR: could not insert module 4.14.71/kernel/fs/nls/nls_ascii.ko.xz: Invalid module format
\end{verbatim}
Easy enough to fix: decompress and re-run \texttt{insmod}:
\begin{verbatim}
xz -d 4.14.71/kernel/fs/nls/nls_ascii.ko.xz
\end{verbatim}
Seems to me rather weird to install a module in a format that can't readily be used, though.

\subsection{2020-01-04}
A writer for ``The Atlantic'' \htmladdnormallink{appears to have mistaken}{https://www.theatlantic.com/technology/archive/2020/01/ice-contract-github-sparks-developer-protests/604339/} a subset of naive progressives within the Open Source community for the Open Source community itself.  There is no ``existential question'' about the nature of Open Source, but there is a question about the judgemental capabilities of those who joined the Open Source bandwagon without actually comprehending what \htmladdnormallink{freedom \#0}{https://www.gnu.org/philosophy/free-sw.html} meant.  Their bubble having been popped, they are now upset at what they \emph{thought} Open Source was and are acting like it's a revelation to anyone but themselves.

\subsection{2020-01-01}
Sonos \htmladdnormallink{considered harmful}{https://twitter.com/atomicthumbs/status/1210662988828442624}.

\subsection{2019-12-18}
New Seasons has \htmladdnormallink{sold}{https://www.wweek.com/news/2019/12/10/portlands-new-seasons-market-sells-to-south-korean-company/} \htmladdnormallink{out}{https://www.wweek.com/news/2019/12/11/portland-labor-group-says-new-seasonss-sale-to-south-korean-company-breaks-a-promise-to-workers/} to South Korea, not seeming to realize that many of us prefer to support our \emph{local} enconomy rather than international business.

\subsection{2019-09-21}
It appears that the \htmladdnormallink{FCC License Manager webpage}{https://wireless2.fcc.gov/UlsEntry/licManager/login.jsp} doesn't work when \texttt{network.http.sendRefererHeader} is set to \texttt{0}.

\subsection{2019-08-27}
\verb|emerge-webrsync: warning: FEATURES=webrsync-gpg is deprecated, see the make.conf(5) man page.|  Hrmph.  After a bit of searching, I found that the global \texttt{FEATURES} flag in \texttt{/etc/make.conf} was migrated to a per-repo setting in \verb|/etc/portage/repos.conf/${NAME}.conf|.  At first I tried using regular \texttt{rsync} by both setting \texttt{sync-type = rsync} and \texttt{sync-rsync-verify-metamanifest = yes} then installing \texttt{app-crypt/openpgp-keys-gentoo-release} and \texttt{app-portage/gemato}, but the cryptographic verification was extremely slow so I moved back to \texttt{webrsync} by setting \texttt{sync-type = webrsync} and \texttt{sync-webrsync-verify-signature = true} instead.

\subsection{2019-07-28}
Pokemon Go \htmladdnormallink{considered harmful}{https://www.networkworld.com/article/3099092/the-cia-nsa-and-pokmon-go.html}.

\subsection{2019-05-26}
I managed to fix the tearing I was experiencing when using \texttt{mplayer} by changing the video output device with \texttt{-vo gl}; the default was \texttt{xv}.  A list of available devices can be gotten by \texttt{-vo help}.  I was also able to change the default by adding \texttt{vo=gl} to \verb|~/.mplayer/config|.

\subsection{2019-05-04}
Workaround for the Firefox add-on signature verification failures: \texttt{xpinstall.signatures.required} to \texttt{false}.  Warning: this is \emph{insecure} and you do so at your own risk.

\subsection{2019-03-30}
\begin{verbatim}
frostsnow@localhost ~/hobby/www.frostsnow.net $ gs
gs: relocation error: /usr/lib/libgnutls.so.30: symbol _idn2_punycode_decode version IDN2_0.0.0 not defined in file libidn2.so.0 with link time reference
\end{verbatim}
I'm not sure what caused this, and recompiling \texttt{ghostscript-gpl} threw a similar linking error (which I forgot to record), but I managed to "fix" it via:
\begin{verbatim}
emerge -avC cups-filters cups
USE="-cups" emerge -1av ghostscript-gpl
\end{verbatim}
I'm\ldots going to hope the problem just stays away now.

\subsection{2019-01-27}
If programs are written by programmers it follows that applications are written by applicationers.

\subsection{2018-12-30}
\begin{quote}
\begin{verbatim}
frostsnow@hesse ~/linux-4.19.13 $ make
Makefile:958: *** "Cannot generate ORC metadata for CONFIG_UNWINDER_ORC=y, please install libelf-dev, libelf-devel or elfutils-libelf-devel".  Stop.
frostsnow@hesse ~/linux-4.19.13 $ grep ORC .config
CONFIG_IRQ_FORCED_THREADING=y
# CONFIG_I2C_NFORCE2 is not set
# CONFIG_UNWINDER_ORC is not set
\end{verbatim}
\end{quote}
After some digging, this appears to be caused by: \htmladdnormallink{https://lkml.org/lkml/2017/12/25/211}{https://lkml.org/lkml/2017/12/25/211}.  So much for breaking my woeful kernel upgrade streak.

\subsection{2018-12-22}
I had an issue where audio stopped working on my Novena.  Looking via '\texttt{alsamixer}' showed that it appeared to be using '\texttt{DW-HDMI}' instead of \texttt{imx-audio-es8328}'.  I was able to get it working again by creating '\texttt{/etc/asound.conf}' with contents:

\begin{quote}
\begin{verbatim}
pcm.!default {
	type hw
	card 1
}
ctl.!default {
	type hw
	card 1
}
\end{verbatim}
\end{quote}
Found at: \htmladdnormallink{https://superuser.com/questions/626606/how-to-make-alsa-pick-a-preferred-sound-device-automatically}{https://superuser.com/questions/626606/how-to-make-alsa-pick-a-preferred-sound-device-automatically}.

\subsection{2018-12-09}
In order to clone from an HTTPS site which uses self-signed certificates, use '\texttt{GIT_SSL_CAINFO="/path/to/selfsigned/cert.pem" git clone blah blah}'.

\subsection{2018-11-03}
I ran into an issue after a system upgrade (Gentoo) last night where X wouldn't start after the upgrade.  The error was: \texttt{(EE) parse_vt_settings: Cannot open /dev/tty0 (Permission denied)}.  It looks like in moving \texttt{x11-base/xorg-server} from version \texttt{1.19.5-r2} to \texttt{1.20.3}  the \texttt{suid} USE flag was disabled; enabling it fixed the issue for me.

\subsection{2018-11-02}
Google Docs \htmladdnormallink{considered harmful}{https://www.eff.org/deeplinks/2018/10/privacy-badger-now-fights-more-sneaky-google-tracking}.

\subsection{2018-10-21}
Noting: In order to upgrade LibreCMC, run \texttt{scp image-sysupgrade.bin root@address:/tmp/} then \texttt{sysupgrade -v image-sysupgrade.bin}.

\subsection{2018-10-20}
Looks like \texttt{stunnel} 5.43 (or perhaps some slightly earlier version) now only binds to the loopback interface by default; as a result I had to change \texttt{accept = 874} to \texttt{accept = 0.0.0.0:874}.

\subsection{2018-10-05}
In a marvelous bit of \htmladdnormallink{Security Theater}{https://en.wikipedia.org/wiki/Security_theater}, Google Voice prevented me from signing in while in a different town, then, upon returning to my usual town, signing in, then telling Google that "Yes, this was me", tells me that, "For your security, we'll continue to show this alert in your Recent security events page." which means that, should someone else \emph{actually} try signing into my account I will not notice because I will assume that the warning is referring to the previous attempt which I have already marked as valid.  Can I disable this garbage, please?

\subsection{2018-10-01}
Managed to fix my XTerm display after the Gentoo \texttt{CHOST} change; the \texttt{.Xresources} file now doesn't like \\
    \texttt{XTerm*background: black} \\
    \texttt{XTerm*foreground: white} \\
\ldots but does accept: \\
    \texttt{xterm*background: black} \\
    \texttt{xterm*background: white} \\
Go figure.

\subsection{2018-04-14}
Noting: Google Voice requires \texttt{media.peerconnection.enabled} to be \texttt{true} in order to display data; likewise, Nextdoor requires \texttt{network.http.sendRefererHeader} to be greater than \texttt{0} in order to log in.

\subsection{2018-04-09}
PKI considered harmful: \htmladdnormallink{https://arstechnica.com/information-technology/2018/03/23000-https-certificates-axed-after-ceo-e-mails-private-keys/}{https://arstechnica.com/information-technology/2018/03/23000-https-certificates-axed-after-ceo-e-mails-private-keys/}.

\subsection{2018-02-25}
TIL \texttt{xclip -selection clipboard} in order to use the \texttt{Ctrl-V} paste buffer (the default selection is \texttt{primary}).

\subsection{2018-02-21}
Every time an employer asks for a "can do" attitude: \htmladdnormallink{https://www.youtube.com/watch?v=CRMcSAgoabw}{https://www.youtube.com/watch?v=CRMcSAgoabw}.

\subsection{2017-12-11}
Web considered harmful: \htmladdnormallink{https://freedom-to-tinker.com/2017/11/15/no-boundaries-exfiltration-of-personal-data-by-session-replay-scripts/}{https://freedom-to-tinker.com/2017/11/15/no-boundaries-exfiltration-of-personal-data-by-session-replay-scripts/}.

\subsection{2017-12-04}
Poem I wrote during a recent fever:
\begin{center}
\begin{verse}
{\large\textbf{The Googlaug}} \\
Through me you pass into the Chamber of Echos, \\
Through me you pass into eternal surveillance, \\
Through me among consumers in debt for aye. \\
Boole the fab of my circuit exec'd, \\
To bootstrap me were the threads of Free Software, \\
Open Source, and Hacker Culture. \\
Before me things centralized were none, save things \\
In meatspace, and in meatspace I endure. \\
Free Will abandon, ye who SYN here.
\end{verse}
\end{center}

\subsection{2017-10-30}
You have died of skyrocketing property values: \htmladdnormallink{http://www.wweek.com/culture/2017/10/30/holy-crap-theres-a-new-oregon-trail-video-game-with-craft-kombucha-and-great-notion-key-lime-pie/}{http://www.wweek.com/culture/2017/10/30/holy-crap-theres-a-new-oregon-trail-video-game-with-craft-kombucha-and-great-notion-key-lime-pie/}

\subsection{2017-10-17}
Web considered harmful: \htmladdnormallink{https://www.eff.org/deeplinks/2017/09/open-letter-w3c-director-ceo-team-and-membership}{https://www.eff.org/deeplinks/2017/09/open-letter-w3c-director-ceo-team-and-membership}.

\subsection{2017-08-16}
A segment from one of my favourite documentaries: \htmladdnormallink{https://www.youtube.com/embed/4xoM6-1SWl4?start=2958&end=3111}{https://www.youtube.com/embed/4xoM6-1SWl4?start=2958&end=3111}.

\subsection{2017-08-04}
\texttt{test a # echo 1 > crlnumber \\
test a # openssl ca -gencrl -config openssl.cnf -cert certs/a.pem -keyfile private/a.pem -out crl/a.pem \\
Using configuration from openssl.cnf \\
unable to load number from .//crlnumber \\
error while loading CRL number \\
140084037076624:error:0D066096:asn1 encoding routines:a2i_ASN1_INTEGER:short line:f_int.c:210: \\
test a # echo 10 > crlnumber \\
test a # openssl ca -gencrl -config openssl.cnf -cert certs/a.pem -keyfile private/a.pem -out crl/a.pem \\
Using configuration from openssl.cnf \\
test a #} \\
Guess one isn't a number anymore.  It must feel lonely.

\subsection{2017-07-10}
\htmladdnormallink{Double King}{https://www.youtube.com/watch?v=w\_MSFkZHNi4} is a pretty neat animated video.

\end{document}
