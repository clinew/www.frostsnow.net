\documentclass{article}

\usepackage{html}

\begin{document}

Thoughts that are not worth their own blog.

\subsection{2020-03-18}
Web \htmladdnormallink{considered harmful}{https://drewdevault.com/2020/03/18/Reckless-limitless-scope.html}.

\subsection{2020-02-03}
Last time I audited my home network with \texttt{nmap}, my scanning box got banned from my server, so this time I used \texttt{fail2ban-client set ssh-iptables addignoreip IP\_ADDRESS} in order to prevent my scanning box from getting banned.

\subsection{2020-01-22}
Got this error when trying to mount my BeagleBone Black's boot partition:
\begin{verbatim}
[883063.268285] FAT-fs (mmcblk0p1): IO charset ascii not found
\end{verbatim}
Turns out the proper kernel module was compressed for some reason and wouldn't load:
\begin{verbatim}
dante /lib/modules # insmod 4.14.71/kernel/fs/nls/nls_ascii.ko.xz
insmod: ERROR: could not insert module 4.14.71/kernel/fs/nls/nls_ascii.ko.xz: Invalid module format
\end{verbatim}
Easy enough to fix: decompress and re-run \texttt{insmod}:
\begin{verbatim}
xz -d 4.14.71/kernel/fs/nls/nls_ascii.ko.xz
\end{verbatim}
Seems to me rather weird to install a module in a format that can't readily be used, though.

\subsection{2020-01-04}
A writer for ``The Atlantic'' \htmladdnormallink{appears to have mistaken}{https://www.theatlantic.com/technology/archive/2020/01/ice-contract-github-sparks-developer-protests/604339/} a subset of naive progressives within the Open Source community for the Open Source community itself.  There is no ``existential question'' about the nature of Open Source, but there is a question about the judgemental capabilities of those who joined the Open Source bandwagon without actually comprehending what \htmladdnormallink{freedom \#0}{https://www.gnu.org/philosophy/free-sw.html} meant.  Their bubble having been popped, they are now upset at what they \emph{thought} Open Source was and are acting like it's a revelation to anyone but themselves.

\subsection{2020-01-01}
Sonos \htmladdnormallink{considered harmful}{https://twitter.com/atomicthumbs/status/1210662988828442624}.

\subsection{2019-12-18}
New Seasons has \htmladdnormallink{sold}{https://www.wweek.com/news/2019/12/10/portlands-new-seasons-market-sells-to-south-korean-company/} \htmladdnormallink{out}{https://www.wweek.com/news/2019/12/11/portland-labor-group-says-new-seasonss-sale-to-south-korean-company-breaks-a-promise-to-workers/} to South Korea, not seeming to realize that many of us prefer to support our \emph{local} enconomy rather than international business.

\subsection{2019-09-21}
It appears that the \htmladdnormallink{FCC License Manager webpage}{https://wireless2.fcc.gov/UlsEntry/licManager/login.jsp} doesn't work when \texttt{network.http.sendRefererHeader} is set to \texttt{0}.

\subsection{2019-08-27}
\verb|emerge-webrsync: warning: FEATURES=webrsync-gpg is deprecated, see the make.conf(5) man page.|  Hrmph.  After a bit of searching, I found that the global \texttt{FEATURES} flag in \texttt{/etc/make.conf} was migrated to a per-repo setting in \verb|/etc/portage/repos.conf/${NAME}.conf|.  At first I tried using regular \texttt{rsync} by both setting \texttt{sync-type = rsync} and \texttt{sync-rsync-verify-metamanifest = yes} then installing \texttt{app-crypt/openpgp-keys-gentoo-release} and \texttt{app-portage/gemato}, but the cryptographic verification was extremely slow so I moved back to \texttt{webrsync} by setting \texttt{sync-type = webrsync} and \texttt{sync-webrsync-verify-signature = true} instead.

\subsection{2019-07-28}
Pokemon Go \htmladdnormallink{considered harmful}{https://www.networkworld.com/article/3099092/the-cia-nsa-and-pokmon-go.html}.

\subsection{2019-05-26}
I managed to fix the tearing I was experiencing when using \texttt{mplayer} by changing the video output device with \texttt{-vo gl}; the default was \texttt{xv}.  A list of available devices can be gotten by \texttt{-vo help}.  I was also able to change the default by adding \texttt{vo=gl} to \verb|~/.mplayer/config|.

\subsection{2019-05-04}
Workaround for the Firefox add-on signature verification failures: \texttt{xpinstall.signatures.required} to \texttt{false}.  Warning: this is \emph{insecure} and you do so at your own risk.

\subsection{2019-03-30}
\begin{verbatim}
frostsnow@localhost ~/hobby/www.frostsnow.net $ gs
gs: relocation error: /usr/lib/libgnutls.so.30: symbol _idn2_punycode_decode version IDN2_0.0.0 not defined in file libidn2.so.0 with link time reference
\end{verbatim}
I'm not sure what caused this, and recompiling \texttt{ghostscript-gpl} threw a similar linking error (which I forgot to record), but I managed to "fix" it via:
\begin{verbatim}
emerge -avC cups-filters cups
USE="-cups" emerge -1av ghostscript-gpl
\end{verbatim}
I'm\ldots going to hope the problem just stays away now.

\subsection{2019-01-27}
If programs are written by programmers it follows that applications are written by applicationers.

\subsection{2018-12-30}
\begin{quote}
\begin{verbatim}
frostsnow@hesse ~/linux-4.19.13 $ make
Makefile:958: *** "Cannot generate ORC metadata for CONFIG_UNWINDER_ORC=y, please install libelf-dev, libelf-devel or elfutils-libelf-devel".  Stop.
frostsnow@hesse ~/linux-4.19.13 $ grep ORC .config
CONFIG_IRQ_FORCED_THREADING=y
# CONFIG_I2C_NFORCE2 is not set
# CONFIG_UNWINDER_ORC is not set
\end{verbatim}
\end{quote}
After some digging, this appears to be caused by: \htmladdnormallink{https://lkml.org/lkml/2017/12/25/211}{https://lkml.org/lkml/2017/12/25/211}.  So much for breaking my woeful kernel upgrade streak.

\subsection{2018-12-22}
I had an issue where audio stopped working on my Novena.  Looking via '\texttt{alsamixer}' showed that it appeared to be using '\texttt{DW-HDMI}' instead of \texttt{imx-audio-es8328}'.  I was able to get it working again by creating '\texttt{/etc/asound.conf}' with contents:

\begin{quote}
\begin{verbatim}
pcm.!default {
	type hw
	card 1
}
ctl.!default {
	type hw
	card 1
}
\end{verbatim}
\end{quote}
Found at: \htmladdnormallink{https://superuser.com/questions/626606/how-to-make-alsa-pick-a-preferred-sound-device-automatically}{https://superuser.com/questions/626606/how-to-make-alsa-pick-a-preferred-sound-device-automatically}.

\subsection{2018-12-09}
In order to clone from an HTTPS site which uses self-signed certificates, use '\texttt{GIT_SSL_CAINFO="/path/to/selfsigned/cert.pem" git clone blah blah}'.

\subsection{2018-11-03}
I ran into an issue after a system upgrade (Gentoo) last night where X wouldn't start after the upgrade.  The error was: \texttt{(EE) parse_vt_settings: Cannot open /dev/tty0 (Permission denied)}.  It looks like in moving \texttt{x11-base/xorg-server} from version \texttt{1.19.5-r2} to \texttt{1.20.3}  the \texttt{suid} USE flag was disabled; enabling it fixed the issue for me.

\subsection{2018-11-02}
Google Docs \htmladdnormallink{considered harmful}{https://www.eff.org/deeplinks/2018/10/privacy-badger-now-fights-more-sneaky-google-tracking}.

\subsection{2018-10-21}
Noting: In order to upgrade LibreCMC, run \texttt{scp image-sysupgrade.bin root@address:/tmp/} then \texttt{sysupgrade -v image-sysupgrade.bin}.

\subsection{2018-10-20}
Looks like \texttt{stunnel} 5.43 (or perhaps some slightly earlier version) now only binds to the loopback interface by default; as a result I had to change \texttt{accept = 874} to \texttt{accept = 0.0.0.0:874}.

\subsection{2018-10-05}
In a marvelous bit of \htmladdnormallink{Security Theater}{https://en.wikipedia.org/wiki/Security_theater}, Google Voice prevented me from signing in while in a different town, then, upon returning to my usual town, signing in, then telling Google that "Yes, this was me", tells me that, "For your security, we'll continue to show this alert in your Recent security events page." which means that, should someone else \emph{actually} try signing into my account I will not notice because I will assume that the warning is referring to the previous attempt which I have already marked as valid.  Can I disable this garbage, please?

\subsection{2018-10-01}
Managed to fix my XTerm display after the Gentoo \texttt{CHOST} change; the \texttt{.Xresources} file now doesn't like \\
    \texttt{XTerm*background: black} \\
    \texttt{XTerm*foreground: white} \\
\ldots but does accept: \\
    \texttt{xterm*background: black} \\
    \texttt{xterm*background: white} \\
Go figure.

\subsection{2018-04-14}
Noting: Google Voice requires \texttt{media.peerconnection.enabled} to be \texttt{true} in order to display data; likewise, Nextdoor requires \texttt{network.http.sendRefererHeader} to be greater than \texttt{0} in order to log in.

\subsection{2018-04-09}
PKI considered harmful: \htmladdnormallink{https://arstechnica.com/information-technology/2018/03/23000-https-certificates-axed-after-ceo-e-mails-private-keys/}{https://arstechnica.com/information-technology/2018/03/23000-https-certificates-axed-after-ceo-e-mails-private-keys/}.

\subsection{2018-02-25}
TIL \texttt{xclip -selection clipboard} in order to use the \texttt{Ctrl-V} paste buffer (the default selection is \texttt{primary}).

\subsection{2018-02-21}
Every time an employer asks for a "can do" attitude: \htmladdnormallink{https://www.youtube.com/watch?v=CRMcSAgoabw}{https://www.youtube.com/watch?v=CRMcSAgoabw}.

\subsection{2017-12-11}
Web considered harmful: \htmladdnormallink{https://freedom-to-tinker.com/2017/11/15/no-boundaries-exfiltration-of-personal-data-by-session-replay-scripts/}{https://freedom-to-tinker.com/2017/11/15/no-boundaries-exfiltration-of-personal-data-by-session-replay-scripts/}.

\subsection{2017-12-04}
Poem I wrote during a recent fever:
\begin{center}
\begin{verse}
{\large\textbf{The Googlaug}} \\
Through me you pass into the Chamber of Echos, \\
Through me you pass into eternal surveillance, \\
Through me among consumers in debt for aye. \\
Boole the fab of my circuit exec'd, \\
To bootstrap me were the threads of Free Software, \\
Open Source, and Hacker Culture. \\
Before me things centralized were none, save things \\
In meatspace, and in meatspace I endure. \\
Free Will abandon, ye who SYN here.
\end{verse}
\end{center}

\subsection{2017-10-30}
You have died of skyrocketing property values: \htmladdnormallink{http://www.wweek.com/culture/2017/10/30/holy-crap-theres-a-new-oregon-trail-video-game-with-craft-kombucha-and-great-notion-key-lime-pie/}{http://www.wweek.com/culture/2017/10/30/holy-crap-theres-a-new-oregon-trail-video-game-with-craft-kombucha-and-great-notion-key-lime-pie/}

\subsection{2017-10-17}
Web considered harmful: \htmladdnormallink{https://www.eff.org/deeplinks/2017/09/open-letter-w3c-director-ceo-team-and-membership}{https://www.eff.org/deeplinks/2017/09/open-letter-w3c-director-ceo-team-and-membership}.

\subsection{2017-08-16}
A segment from one of my favourite documentaries: \htmladdnormallink{https://www.youtube.com/embed/4xoM6-1SWl4?start=2958&end=3111}{https://www.youtube.com/embed/4xoM6-1SWl4?start=2958&end=3111}.

\subsection{2017-08-04}
\texttt{test a # echo 1 > crlnumber \\
test a # openssl ca -gencrl -config openssl.cnf -cert certs/a.pem -keyfile private/a.pem -out crl/a.pem \\
Using configuration from openssl.cnf \\
unable to load number from .//crlnumber \\
error while loading CRL number \\
140084037076624:error:0D066096:asn1 encoding routines:a2i_ASN1_INTEGER:short line:f_int.c:210: \\
test a # echo 10 > crlnumber \\
test a # openssl ca -gencrl -config openssl.cnf -cert certs/a.pem -keyfile private/a.pem -out crl/a.pem \\
Using configuration from openssl.cnf \\
test a #} \\
Guess one isn't a number anymore.  It must feel lonely.

\subsection{2017-07-10}
\htmladdnormallink{Double King}{https://www.youtube.com/watch?v=w\_MSFkZHNi4} is a pretty neat animated video.

\end{document}
